% Options for packages loaded elsewhere
\PassOptionsToPackage{unicode}{hyperref}
\PassOptionsToPackage{hyphens}{url}
%
\documentclass[
]{article}
\usepackage{lmodern}
\usepackage{amssymb,amsmath}
\usepackage{ifxetex,ifluatex}
\ifnum 0\ifxetex 1\fi\ifluatex 1\fi=0 % if pdftex
  \usepackage[T1]{fontenc}
  \usepackage[utf8]{inputenc}
  \usepackage{textcomp} % provide euro and other symbols
\else % if luatex or xetex
  \usepackage{unicode-math}
  \defaultfontfeatures{Scale=MatchLowercase}
  \defaultfontfeatures[\rmfamily]{Ligatures=TeX,Scale=1}
\fi
% Use upquote if available, for straight quotes in verbatim environments
\IfFileExists{upquote.sty}{\usepackage{upquote}}{}
\IfFileExists{microtype.sty}{% use microtype if available
  \usepackage[]{microtype}
  \UseMicrotypeSet[protrusion]{basicmath} % disable protrusion for tt fonts
}{}
\makeatletter
\@ifundefined{KOMAClassName}{% if non-KOMA class
  \IfFileExists{parskip.sty}{%
    \usepackage{parskip}
  }{% else
    \setlength{\parindent}{0pt}
    \setlength{\parskip}{6pt plus 2pt minus 1pt}}
}{% if KOMA class
  \KOMAoptions{parskip=half}}
\makeatother
\usepackage{xcolor}
\IfFileExists{xurl.sty}{\usepackage{xurl}}{} % add URL line breaks if available
\IfFileExists{bookmark.sty}{\usepackage{bookmark}}{\usepackage{hyperref}}
\hypersetup{
  pdftitle={Data Transformation with dplyr},
  pdfauthor={Karen Chu Sam},
  hidelinks,
  pdfcreator={LaTeX via pandoc}}
\urlstyle{same} % disable monospaced font for URLs
\usepackage[margin=1in]{geometry}
\usepackage{color}
\usepackage{fancyvrb}
\newcommand{\VerbBar}{|}
\newcommand{\VERB}{\Verb[commandchars=\\\{\}]}
\DefineVerbatimEnvironment{Highlighting}{Verbatim}{commandchars=\\\{\}}
% Add ',fontsize=\small' for more characters per line
\usepackage{framed}
\definecolor{shadecolor}{RGB}{248,248,248}
\newenvironment{Shaded}{\begin{snugshade}}{\end{snugshade}}
\newcommand{\AlertTok}[1]{\textcolor[rgb]{0.94,0.16,0.16}{#1}}
\newcommand{\AnnotationTok}[1]{\textcolor[rgb]{0.56,0.35,0.01}{\textbf{\textit{#1}}}}
\newcommand{\AttributeTok}[1]{\textcolor[rgb]{0.77,0.63,0.00}{#1}}
\newcommand{\BaseNTok}[1]{\textcolor[rgb]{0.00,0.00,0.81}{#1}}
\newcommand{\BuiltInTok}[1]{#1}
\newcommand{\CharTok}[1]{\textcolor[rgb]{0.31,0.60,0.02}{#1}}
\newcommand{\CommentTok}[1]{\textcolor[rgb]{0.56,0.35,0.01}{\textit{#1}}}
\newcommand{\CommentVarTok}[1]{\textcolor[rgb]{0.56,0.35,0.01}{\textbf{\textit{#1}}}}
\newcommand{\ConstantTok}[1]{\textcolor[rgb]{0.00,0.00,0.00}{#1}}
\newcommand{\ControlFlowTok}[1]{\textcolor[rgb]{0.13,0.29,0.53}{\textbf{#1}}}
\newcommand{\DataTypeTok}[1]{\textcolor[rgb]{0.13,0.29,0.53}{#1}}
\newcommand{\DecValTok}[1]{\textcolor[rgb]{0.00,0.00,0.81}{#1}}
\newcommand{\DocumentationTok}[1]{\textcolor[rgb]{0.56,0.35,0.01}{\textbf{\textit{#1}}}}
\newcommand{\ErrorTok}[1]{\textcolor[rgb]{0.64,0.00,0.00}{\textbf{#1}}}
\newcommand{\ExtensionTok}[1]{#1}
\newcommand{\FloatTok}[1]{\textcolor[rgb]{0.00,0.00,0.81}{#1}}
\newcommand{\FunctionTok}[1]{\textcolor[rgb]{0.00,0.00,0.00}{#1}}
\newcommand{\ImportTok}[1]{#1}
\newcommand{\InformationTok}[1]{\textcolor[rgb]{0.56,0.35,0.01}{\textbf{\textit{#1}}}}
\newcommand{\KeywordTok}[1]{\textcolor[rgb]{0.13,0.29,0.53}{\textbf{#1}}}
\newcommand{\NormalTok}[1]{#1}
\newcommand{\OperatorTok}[1]{\textcolor[rgb]{0.81,0.36,0.00}{\textbf{#1}}}
\newcommand{\OtherTok}[1]{\textcolor[rgb]{0.56,0.35,0.01}{#1}}
\newcommand{\PreprocessorTok}[1]{\textcolor[rgb]{0.56,0.35,0.01}{\textit{#1}}}
\newcommand{\RegionMarkerTok}[1]{#1}
\newcommand{\SpecialCharTok}[1]{\textcolor[rgb]{0.00,0.00,0.00}{#1}}
\newcommand{\SpecialStringTok}[1]{\textcolor[rgb]{0.31,0.60,0.02}{#1}}
\newcommand{\StringTok}[1]{\textcolor[rgb]{0.31,0.60,0.02}{#1}}
\newcommand{\VariableTok}[1]{\textcolor[rgb]{0.00,0.00,0.00}{#1}}
\newcommand{\VerbatimStringTok}[1]{\textcolor[rgb]{0.31,0.60,0.02}{#1}}
\newcommand{\WarningTok}[1]{\textcolor[rgb]{0.56,0.35,0.01}{\textbf{\textit{#1}}}}
\usepackage{graphicx,grffile}
\makeatletter
\def\maxwidth{\ifdim\Gin@nat@width>\linewidth\linewidth\else\Gin@nat@width\fi}
\def\maxheight{\ifdim\Gin@nat@height>\textheight\textheight\else\Gin@nat@height\fi}
\makeatother
% Scale images if necessary, so that they will not overflow the page
% margins by default, and it is still possible to overwrite the defaults
% using explicit options in \includegraphics[width, height, ...]{}
\setkeys{Gin}{width=\maxwidth,height=\maxheight,keepaspectratio}
% Set default figure placement to htbp
\makeatletter
\def\fps@figure{htbp}
\makeatother
\setlength{\emergencystretch}{3em} % prevent overfull lines
\providecommand{\tightlist}{%
  \setlength{\itemsep}{0pt}\setlength{\parskip}{0pt}}
\setcounter{secnumdepth}{-\maxdimen} % remove section numbering

\title{Data Transformation with dplyr}
\author{Karen Chu Sam}
\date{2020-03-27T21:13:14-05:00}

\begin{document}
\maketitle

So it's been about more than one month since I last published in the
blog. The lockdown during this coronavirus has kept me a little bit busy
emotionally, but now I've learned to live with all the coronavirus news
going around and to enjoy the current state of the lockdown.

In the last month, I've been juggling between Statistical Inference from
the Data Science specialization in \emph{Coursera} and learning how to
use the dplyr package by following \emph{R for Data Science} (Wickham \&
Grolemund, 2016). The book is pretty good to learn about the tools for
data science. Furthermore, the book provides a good amount of exercises
to practice and it complements the course on \emph{Coursera}. I can
strongly recommend doing the exercises in the book because that's how
you internalize and digest the things learned.

In this post, I will explain about what I've learned with the dplyr
package and how is the \emph{Statistical Inference} course.

\hypertarget{dplyr}{%
\section{Dplyr}\label{dplyr}}

We will need the both dplyr and the nycflights13 package. The
nycflights13 package contains information abouth flights departing from
New York in 2013. We will use data tables from this package to perform
data transformation with dplyr. The dlyr package contains a set of
useful functions to perform the most common data transformation.

\begin{Shaded}
\begin{Highlighting}[]
\KeywordTok{library}\NormalTok{(dplyr)}
\end{Highlighting}
\end{Shaded}

\begin{verbatim}
## 
## Attaching package: 'dplyr'
\end{verbatim}

\begin{verbatim}
## The following objects are masked from 'package:stats':
## 
##     filter, lag
\end{verbatim}

\begin{verbatim}
## The following objects are masked from 'package:base':
## 
##     intersect, setdiff, setequal, union
\end{verbatim}

\begin{Shaded}
\begin{Highlighting}[]
\KeywordTok{library}\NormalTok{(nycflights13)}
\end{Highlighting}
\end{Shaded}

Note that the flights dataframe is a tibble. This means that\ldots.

dplyr package

dlyr basics: (i)\texttt{filter()}, (ii) arrange(), (iii) select(), (iv)
mutate(), (v) summarize() and pipeline operator

\hypertarget{filter}{%
\subsubsection{\texorpdfstring{\texttt{filter()}}{filter()}}\label{filter}}

\texttt{filter()} allows you to filter rows based on their values. So
let's say you want to filter all flights that either departed or arrived
on women's international day, March 8th. filter(flights, month==3,
day==8) You can also use logical operators in the functions. Let's say
you want to filter flights that departed with more than 1 hour delay.
filter(flights, ( arr\_delay\textgreater=60)) x \%in\% select every row
where x is one of the values in y.

\hypertarget{arrange}{%
\subsubsection{\texorpdfstring{\texttt{arrange()}}{arrange()}}\label{arrange}}

arrange allows you to arrange the rows of the dataframe as you would
like. So, let's say you want to arrange the flights with a descending
dep\_time, then you would have to use arrange(flights, desc(dep\_time)).

\hypertarget{select}{%
\subsubsection{\texorpdfstring{\texttt{select()}}{select()}}\label{select}}

Let's you select columns. So, you're subsetting the dataframe and
selecting only the variables you're interested in.Say you want to select
the tail number and the carrier of the flights dataframe.
select(flights, tail\_num, carrier) Other options useful to know in
select, when you want to select a couple of columns is select(flights,
(year:day), -(carrier:air\_time), everything())

\hypertarget{mutate}{%
\subsection{mutate()}\label{mutate}}

Mutate allows you to add new columns to the dataframe. You will always
see the new variables or columnes at the end of the dataframe. Following
the book, I'll just add two columns. mutate(flights, gain=arr\_delay -
dep\_delay, speed=distance/air\_time*60)

\hypertarget{summarize}{%
\subsection{summarize()}\label{summarize}}

let's say I want to know the average arrival delay of the flights.
summarize(flights, delay=mean(arr\_delay, na.rm=TRUE)) This results in
the mean of the whole column arr\_delay. How about if I want to know the
average arrival delay per day? I'll have to use the function group\_by.

\begin{Shaded}
\begin{Highlighting}[]
\NormalTok{by_day <-}\StringTok{ }\KeywordTok{group_by}\NormalTok{(flights, year, month, day)}
\KeywordTok{summarize}\NormalTok{(by_day, }\DataTypeTok{delay=}\KeywordTok{mean}\NormalTok{(dep_delay, }\DataTypeTok{na.rm=}\OtherTok{TRUE}\NormalTok{))}
\end{Highlighting}
\end{Shaded}

\begin{verbatim}
## # A tibble: 365 x 4
## # Groups:   year, month [12]
##     year month   day delay
##    <int> <int> <int> <dbl>
##  1  2013     1     1 11.5 
##  2  2013     1     2 13.9 
##  3  2013     1     3 11.0 
##  4  2013     1     4  8.95
##  5  2013     1     5  5.73
##  6  2013     1     6  7.15
##  7  2013     1     7  5.42
##  8  2013     1     8  2.55
##  9  2013     1     9  2.28
## 10  2013     1    10  2.84
## # ... with 355 more rows
\end{verbatim}

What I am doing with group\_by() is grouping the flights dataframe by
the 3 columns. But note that it doesn't change how the data looks. So,
if we call by\_day, we'll see the same dataframe.

\begin{Shaded}
\begin{Highlighting}[]
\NormalTok{by_day}
\end{Highlighting}
\end{Shaded}

\begin{verbatim}
## # A tibble: 336,776 x 19
## # Groups:   year, month, day [365]
##     year month   day dep_time sched_dep_time dep_delay arr_time sched_arr_time
##    <int> <int> <int>    <int>          <int>     <dbl>    <int>          <int>
##  1  2013     1     1      517            515         2      830            819
##  2  2013     1     1      533            529         4      850            830
##  3  2013     1     1      542            540         2      923            850
##  4  2013     1     1      544            545        -1     1004           1022
##  5  2013     1     1      554            600        -6      812            837
##  6  2013     1     1      554            558        -4      740            728
##  7  2013     1     1      555            600        -5      913            854
##  8  2013     1     1      557            600        -3      709            723
##  9  2013     1     1      557            600        -3      838            846
## 10  2013     1     1      558            600        -2      753            745
## # ... with 336,766 more rows, and 11 more variables: arr_delay <dbl>,
## #   carrier <chr>, flight <int>, tailnum <chr>, origin <chr>, dest <chr>,
## #   air_time <dbl>, distance <dbl>, hour <dbl>, minute <dbl>, time_hour <dttm>
\end{verbatim}

When I was learning group\_by(), I didn't know about this fact and I
couldn't understand what was so special about this group\_by(). Then,
when we call the summarize() funciton on the grouped data, by\_day,
we'll see that the mean is being calculated per day. We can also
calcualte the average arrival delay per month. For this, we would have
to group the data by year and month only.

\begin{Shaded}
\begin{Highlighting}[]
\NormalTok{by_month <-}\StringTok{ }\KeywordTok{group_by}\NormalTok{(flights, year, month)}
\KeywordTok{summarize}\NormalTok{(by_month, }\DataTypeTok{delay=}\KeywordTok{mean}\NormalTok{(arr_delay, }\DataTypeTok{na.rm=}\OtherTok{TRUE}\NormalTok{))}
\end{Highlighting}
\end{Shaded}

\begin{verbatim}
## # A tibble: 12 x 3
## # Groups:   year [1]
##     year month  delay
##    <int> <int>  <dbl>
##  1  2013     1  6.13 
##  2  2013     2  5.61 
##  3  2013     3  5.81 
##  4  2013     4 11.2  
##  5  2013     5  3.52 
##  6  2013     6 16.5  
##  7  2013     7 16.7  
##  8  2013     8  6.04 
##  9  2013     9 -4.02 
## 10  2013    10 -0.167
## 11  2013    11  0.461
## 12  2013    12 14.9
\end{verbatim}

Now that we have a basic kwowledge of summarize works with group\_by, it
is useful to learn the Pipe operator or also \%\textgreater\%. First of
all, a shortcut for the pipe is ctrl + shift + m. We observe that for
using the summarize function in combination with group\_by(), we have
the first group the dataframe and save it in a new variable. Then, we
proceed to use the new variable in the summarize function(). But that
just takes a lot of time. So instead, we can combine both code lines
with the pipe operator:

\begin{Shaded}
\begin{Highlighting}[]
\NormalTok{flights }\OperatorTok\StringTok{ }
\StringTok{    }\KeywordTok{group_by}\NormalTok{(year, month, day) }\OperatorTok\StringTok{ }
\StringTok{    }\KeywordTok{summarize}\NormalTok{(}\DataTypeTok{mean=}\KeywordTok{mean}\NormalTok{(arr_delay, }\DataTypeTok{na.rm=}\OtherTok{TRUE}\NormalTok{))}
\end{Highlighting}
\end{Shaded}

\begin{verbatim}
## # A tibble: 365 x 4
## # Groups:   year, month [12]
##     year month   day   mean
##    <int> <int> <int>  <dbl>
##  1  2013     1     1 12.7  
##  2  2013     1     2 12.7  
##  3  2013     1     3  5.73 
##  4  2013     1     4 -1.93 
##  5  2013     1     5 -1.53 
##  6  2013     1     6  4.24 
##  7  2013     1     7 -4.95 
##  8  2013     1     8 -3.23 
##  9  2013     1     9 -0.264
## 10  2013     1    10 -5.90 
## # ... with 355 more rows
\end{verbatim}

The pipe operator allows us to write everything in one section.

INCLUDE THE CHEATSHEET FOR DPLYR

\end{document}
